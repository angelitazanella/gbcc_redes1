\documentclass[a4paper,11pt]{modelo/prova}
\usepackage[utf8x]{inputenc}
\usepackage[T1]{fontenc}
\usepackage[brazilian,english]{babel}
%\usepackage[alf]{abntcite} % ABNT
\usepackage{multirow}
\usepackage{array}
\usepackage{graphicx}
\usepackage{hyperref}
\usepackage{enumerate}
\usepackage{multicol}

%===================================
% Cabeçalho
%===================================
\curso{Curso de Bacharelado em Ciência da Computação}
\prova{Avaliação - Fundamentos de Redes}
\disciplina{Redes de Computadores I}

\orientacoes{
1. A prova é individual e sem consulta. \\
2. Não é permitido o uso de qualquer dispositivo eletrônico durante a prova. Caso você seja flagrado utilizando qualquer dispositivo, ele será recolhido e sua prova será zerada. \\
3. Leia a prova com atenção antes de responder. A interpretação das questões faz parte da avaliação. \\
4. As questões objetivas devem ser respondidas com caneta azul ou preta. \\
5. As questões discursivas podem ser respondidas à lápis. Questões respondidas a lápis não poderão ser revisadas futuramente. \\
6. Responda de forma clara e concisa, focando na precisão e relevância das informações. \\
7. Cada questão possui indicação de sua pontuação. \\[.5em] \textbf{Boa prova!}
}

\begin{document}

\maketitle

\begin{enumerate}
   \item (2,0) Imagine que você está configurando uma rede doméstica em um sobrado de dois andares para uma família de quatro pessoas. Os requisitos incluem: trabalho remoto, streaming de vídeos, jogos online e múltiplos dispositivos.

   \textbf{Problema:} Explique como você configuraria a rede nesse sobrado, explicando a diferença entre usar cabos Ethernet e Wi-Fi para conectar os dispositivos à internet. Compare as vantagens e desvantagens de cada uma das tecnologias em termos de largura de banda e interferência.
   
   \textbf{Análise:} Sua resposta deve incluir
   \begin{itemize}
      \item Informações sobre latência, largura de banda e interferências;
      \item Se você optar por uma configuração usando tecnologia Ethernet, informe quais dispositivos se beneficiariam mais dessa conexão. Quais os benefícios e desafios dessa implementação?
      \item Se você optar por uma configuração usando tecnologia Wi-Fi, discorra sobre o posicionamento do roteador wifi/ponto de acesso para otimizar o alcance e a cobertura. Quais os benefícios e desafios dessa implementação?
      \item Se você optar por uma solução híbrida (Ethernet + Wi-Fi), justifique sua escolha com base nos conceitos de largura de banda e interferência. Apresente ainda informações solicitadas nos itens anteriores.
   \end{itemize}

   \item (2,0) Uma empresa de médio porte, chamada ``AngelCorp'', está migrando sua infraestrutura de rede para uma arquitetura mais moderna para suportar o crescimento de suas operações e a adoção de novas tecnologias. A AngelCorp possui escritórios em diferentes cidades, um data center centralizado e um mix de tráfego em tempo real (voz e vídeo) e tráfego não sensível ao tempo (transferência de arquivos, e-mail).

   \textbf{Problema:} Explique o conceito de comutação em redes de computadores. Qual é a importância da comutação para o funcionamento eficiente de uma rede? Compare os métodos de comutação de circuitos e comutação de pacotes, destacando suas principais características, vantagens e desvantagens.

   Em que cenários cada método seria mais adequado para a AngelCorp? Considere os seguintes aspectos:
   \begin{itemize}
      \item Comutação de Circuitos: Em quais situações a comutação de circuitos seria vantajosa para a AngelCorp, considerando o tipo de tráfego que ela precisa suportar? Como ela poderia ser implementada para atender às necessidades da empresa
      \item Comutação de Pacotes: Em quais situações a comutação de pacotes seria mais adequada para a AngelCorp, considerando o tipo de tráfego que ela precisa suportar? Como ela poderia ser utilizada para otimizar o desempenho da rede e garantir a qualidade da experiência do usuário?
   \end{itemize}

   \item (2,0) Imagine que você é o administrador de rede de uma universidade. Durante o período de provas, a rede Wi-Fi do campus fica extremamente congestionada. Muitos alunos reclamam de lentidão ao acessar materiais de estudo online, enviar trabalhos e participar de videoconferências.

   \textbf{Problema:} Explique os conceitos de atraso (delay), perda (loss) e vazão (throughput) em redes de computadores. Como esses fatores interagem entre si em uma rede congestionada como a da universidade?

   Dê exemplos práticos de como o congestionamento impacta a experiência dos alunos ao:
   \begin{itemize}
      \item Acessar vídeos de aulas gravadas, considerando o conceito de vazão.
      \item Participar de sessões de estudo online, considerando o conceito de atraso.
      \item Enviar trabalhos e provas pela internet, considerando o conceito de perda.
   \end{itemize}

   Quais medidas você poderia adotar como administrador de rede para mitigar o congestionamento e melhorar a qualidade da experiência dos alunos durante o período de provas?

   \item (2,0) Uma empresa de desenvolvimento de software, ``AngelCorp'', possui dois escritórios, um em São Paulo (Brasil) e outro em Curitiba (Brasil). Para facilitar a colaboração entre as equipes, os escritórios estão conectados por um link de comunicação de alta velocidade que passa por alguns equipamentos intermediários.
   
   \textbf{Dados:}
   \begin{itemize}
      \item A distância entre os escritórios é de aproximadamente 400 km.
      \item A velocidade de propagação do sinal no meio físico (fibra óptica) é de $2 * 10^8 m/s$.
      \item A capacidade do link de comunicação é de 1 Gbps.
      \item O caminho entre os escritórios inclui 3 roteadores intermediários.
      \item Cada roteador introduz um atraso de processamento de 1 ms por pacote.
   \end{itemize}

   \textbf{Problema:} 
   \begin{enumerate}
      \item Um desenvolvedor em São Paulo precisa enviar um arquivo de código-fonte de 4 MB para um colega em Curitiba. Calcule:
      \begin{itemize}
         \item O atraso de propagação (propagation delay).
         \item O atraso de transmissão (transmission delay).
         \item O atraso de processamento total (considerando os 3 roteadores).
         \item O atraso total (soma de todos os atrasos).
      \end{itemize}
      Com base nos seus cálculos, explique como a distância e o número de roteadores intermediários afetam o atraso total na comunicação entre os escritórios, conforme discutido no Capítulo 1 do Kurose.
   \end{enumerate}

   \item (2,0) Descreva detalhadamente o Modelo OSI e a arquitetura TCP/IP. Explique as funções de cada camada em ambos os modelos e compare suas principais diferenças, conforme discutido no Capítulo 1 do Kurose. Dê exemplos de como os protocolos HTTP, TCP e IP se encaixam nessas arquiteturas e como eles são utilizados na comunicação entre um navegador web e um servidor web.
   \end{enumerate}

\end{document}
