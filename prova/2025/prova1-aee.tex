\documentclass[a4paper,11pt]{modelo/prova}
\usepackage[utf8x]{inputenc}
\usepackage[T1]{fontenc}
\usepackage[brazilian,english]{babel}
%\usepackage[alf]{abntcite} % ABNT
\usepackage{multirow}
\usepackage{array}
\usepackage{graphicx}
\usepackage{hyperref}
\usepackage{enumerate}
\usepackage{multicol}

%===================================
% Cabeçalho
%===================================
\curso{Curso de Bacharelado em Ciência da Computação}
\prova{Avaliação - Fundamentos de Redes}
\disciplina{Redes de Computadores I}

\orientacoes{
1. A prova é individual e sem consulta. \\
2. Não é permitido o uso de qualquer dispositivo eletrônico durante a prova. Caso você seja flagrado utilizando qualquer dispositivo, ele será recolhido e sua prova será zerada. \\
3. Leia a prova com atenção antes de responder. A interpretação das questões faz parte da avaliação. \\
4. As questões objetivas devem ser respondidas com caneta azul ou preta. \\
5. As questões discursivas podem ser respondidas à lápis. Questões respondidas a lápis não poderão ser revisadas futuramente. \\
6. Responda de forma clara e concisa, focando na precisão e relevância das informações. \\
7. Cada questão possui indicação de sua pontuação. \\[.5em] \textbf{Boa prova!}
}

\begin{document}

\maketitle

\begin{enumerate}
   \item (2,0) Imagine que você está configurando uma rede doméstica em um sobrado de dois andares para uma família de quatro pessoas. Os requisitos incluem: trabalho remoto, streaming de vídeos, jogos online e múltiplos dispositivos.

   \textbf{Problema:} Qual a diferença entre usar cabos de rede (Ethernet) e Wi-Fi para ligar os aparelhos na internet?
   
   \textbf{Análise:}
   \begin{enumerate}[a)]
      \item Como você faria para usar cabos de rede em casa? Quais aparelhos você ligaria com cabos? Quais as vantagens e dificuldades disso?
      \item Como você faria para usar o Wi-Fi em casa? Onde você colocaria o roteador para o Wi-Fi pegar bem em todos os lugares? Quais as vantagens e dificuldades disso?
      \item Qual a melhor forma de usar os dois (cabos e Wi-Fi) juntos para atender a família? Por que você acha que essa é a melhor opção?
   \end{enumerate}

   \item (2,0) Uma empresa chamada AngelCorp está mudando a rede de computadores para ficar mais moderna. A empresa tem escritórios em várias cidades, um data center e precisa de internet rápida para voz, vídeo e arquivos.

   \textbf{Problema:} O que é ``comutação'' em redes de computadores? Por que ela é importante?

   Compare os dois tipos de comutação:
   \begin{itemize}
      \item Comutação de Circuitos: Como seria usar esse tipo de comutação na AngelCorp? Em quais situações ele seria bom?
      \item Como seria usar esse tipo de comutação na AngelCorp? Em quais situações ele seria melhor?
   \end{itemize}

   \item (2,0) Você é o responsável pela rede de internet de uma universidade. Na época de provas, a internet fica muito lenta. Os alunos reclamam que não conseguem acessar os materiais de estudo, enviar trabalhos e assistir às aulas online.

   \textbf{Problema:} Explique o que são ``atraso'', ``perda'' e ``vazão'' em redes de computadores. Como eles afetam a internet quando ela está congestionada?

   Dê exemplos de como a internet lenta atrapalha os alunos quando eles precisam:
   \begin{itemize}
      \item Assistir vídeos de aulas.
      \item Participar de aulas online.
      \item Enviar trabalhos e provas.

      O que você faria para melhorar a internet na época de provas?

   \item (2,0) A empresa AngelCorp tem dois escritórios, um em São Paulo e outro em Curitiba. Os escritórios estão ligados por uma conexão rápida.
   
   \textbf{Dados:}
   \begin{itemize}
      \item A distância entre os escritórios é de aproximadamente 400 km.
      \item A velocidade de propagação do sinal no meio físico (fibra óptica) é de $2 * 10^8 m/s$.
      \item A conexão tem capacidade de 1 1 Gbps.
      \item No caminho entre os escritórios, há 3 roteadores.
      \item Cada roteador atrasa 1 ms por pacote.
   \end{itemize}

   \textbf{Problema:} 
   \begin{enumerate}
      \item Um programador em São Paulo precisa enviar um arquivo de 4 MB para um colega em Curitiba. Calcule:
      \begin{itemize}
         \item O atraso de propagação (propagation delay).
         \item O atraso de transmissão (transmission delay).
         \item O atraso de processamento total (considerando os 3 roteadores).
         \item O atraso total (soma de todos os atrasos).
      \end{itemize}
      Explique como a distância e os roteadores afetam o tempo que leva para o arquivo chegar, usando o que você aprendeu no Capítulo 1 do livro do Kurose.
   \end{enumerate}

   \item (2,0) Descreva o Modelo OSI e a arquitetura TCP/IP. Explique o que cada camada faz e compare os dois modelos. Use exemplos de como os protocolos HTTP, TCP e IP funcionam juntos quando você acessa um site.
   \end{enumerate}

\end{document}
